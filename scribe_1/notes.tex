\documentclass[usletter]{article}
\usepackage{graphicx}
\usepackage{amsfonts}
\usepackage{amsthm}
\usepackage{amsmath}
\usepackage{amssymb}
\usepackage{scribe}
\usepackage[margin=1.5in]{geometry}

\begin{document}

\makeheader{Lowell Bander}{January 4, 2016}{1}{Deterministic Computation}

During the first lecture of the course, we covered the formalisms for and semantics of the Turing Machine (TM), a universal model of computation. We then proceeded to describe a TM which checks whether its input is a palindrome, so as to demonstrate how to construct a TM to solve an arbitrary problem.

During the second half of the lecture, we covered the notion of efficiency as it pertains to the time a TM takes to compute its output. Lastly, we showed that our standard definition of a TM can be used to simulate more complicated TM's, such as those with an enriched alphabet and those with bidirectional tapes. This last bit comes back to the notion that the Turing Machine is in fact a universal model of computation -- it can compute anything which is computable, even things which may seem to be computable only with more complex models of computation.


\end{document}
